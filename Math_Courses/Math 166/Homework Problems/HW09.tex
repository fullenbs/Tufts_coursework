\documentclass [12pt] {article}
\usepackage[dvips]{graphicx}
\usepackage{color}
\usepackage{amssymb,amsmath,amsfonts,amsthm,bm}
\usepackage{xfrac}
\usepackage{html}
\usepackage[backref,letterpaper]{hyperref}

\textheight 9.0in
\textwidth 6.5in
\oddsidemargin -0.225in
\evensidemargin -0.225in
\topmargin -0.5in
%\baselineskip=10pt

\newcommand{\bfdelta}{{\bm{\delta}}}
\newcommand{\bfnu}{{\bm{\nu}}}
\newcommand{\bff}{{\bm{f}}}
\newcommand{\bbC}{{\mathbb{C}}}
\newcommand{\calD}{{\mathcal{D}}}
\newcommand{\bbF}{{\mathbb{F}}}
\newcommand{\calL}{{\mathcal{L}}}
\newcommand{\bbR}{{\mathbb{R}}}
\newcommand{\bfP}{{\mathbf{P}}}
\newcommand{\calS}{{\mathcal{S}}}
\newcommand{\bfu}{{\mathbf{u}}}
\newcommand{\bfv}{{\mathbf{v}}}
\newcommand{\bfw}{{\mathbf{w}}}
\newcommand{\bfx}{{\mathbf{x}}}
\newcommand{\bfy}{{\mathbf{y}}}
\newcommand{\bfz}{{\mathbf{z}}}
\newcommand{\bbZ}{{\mathbb{Z}}}
\newcommand{\bfzero}{{\mathbf{0}}}
\newcommand{\bfone}{{\mathbf{1}}}
\newcommand{\Var}{{\mbox{Var}}}
\newcommand{\notsubseteq}{{\subseteq \hspace{-0.15in}/\;}}
\newcommand{\nin}{\notin}
\newcommand{\addlink}[2]{{\htmladdnormallink {\textcolor{blue}{{#1}}}{{#2}}}}
\newcommand{\Prob}{{\mbox{Prob}}}
\newcommand{\supp}{{\mbox{supp}}}

\begin{document}

\thispagestyle{empty}
\begin{center}
Tufts University\\
Department of Mathematics\\
Spring 2022 \\
\end{center}
\begin{center}
{\bf MA 166: Statistics}\\
\end{center}
\begin{center}
{\bf\Large Homework 9 {\small (v1.1)}}~\footnote{\copyright 2022, Bruce M. Boghosian, all rights reserved.}\\
Assigned Monday 11 April 2022\\
Due Tuesday 19 April 2022 at 11:59 pm EDT.
\end{center}

\begin{enumerate}

\item Larsen \& Marx, Problem 11.2.20, p. 541

\item Larsen \& Marx, Problem 11.2.28, p. 543

\item Larsen \& Marx, Problem 11.3.6, p. 556

\item Larsen \& Marx, Problem 11.3.10, p. 556

\item You have reason to believe that the \textit{positive quantities} $x$ and $y$ are related by an equation of the form
\[
e^{a+bx} + e^{-cy} = 1.
\]
	\begin{enumerate}
	\item Prove that $c$ must be positive.
	\item Suppose that you have $n$ pairs of \textit{positive numbers}, $(x_i,y_i)$ for $i=1,\ldots,n$, and you would like to find the best fit for all three parameters $a$, $b$ and $c$.  You might think that this would require solving three simultaneous nonlinear equations in three unknowns.  Find a way to do it that involves solving only one nonlinear equation for the unknown parameter $c$.
	\end{enumerate}

\end{enumerate}

\end{document}
